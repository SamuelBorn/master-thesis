\chapter{Our Approach}
\label{ch:approach}

\section{Planarity}
\label{sec:approach:planarity}

\todo[inline]{
    always intertial flow cutter to compute unless used otherwise
}

\todo[inline]{
    street graphs are almost planar
    planar graphs have 2/3 balanced separators of size O(sqrt(n))
    cite paper
    worse than road graphs
    how do road grpahs behave when planarized
    
    interpret all roads as linear segments
    for every intersection add new vertex
    replace old edge
    to make efficiently: spatial index with bounding box of every edge
    figure: explanation
    query intersections of bounding boxes and then check for intersection
    multiple intersections possible
    sort points add to graph
    psuedo code

    karlsruhe 100k nodes 2.5k intersections
    germany 5m nodes 100k intersctions
    bit more than sqrt(n) intersections that were mentioned in other papers but still in the same range

    separator sizes did not change much
    2/3 balanced with cbrt(n)
    figure: plot germany planar vs non planar

    separators of non planar are often seps of planar or can be extended to be one with just a few nodes
    figure: map karlruhe non planar extendend to 


    the findings from above leads us to assume the just almost planarity of road networks no impact on sep size
    and look mostly on planar graphs
}
