%% LaTeX2e class for student theses
%% sections/evaluation.tex

\chapter{Example}
\label{ch:Example}%anchor point for a reference
In this chapter, we give you some advice for \LaTeX{} and scientific writing to avoid common mistakes.
Most of these rules are just good practice and are not necessarily true in every situation but following them will not hurt you.
You should already be comfortable with \LaTeX{} and Git, take a look at this beginners guide \url{https://www.overleaf.com/learn} if you are not.


\section{Chapters, Sections and Subsections}
\label{sec:Example:Chapters}
Your text should be partitioned by the commands "\texttt{\textbackslash{}chapter}", "\texttt{\textbackslash{}section}", and "\texttt{\textbackslash{}subsection}".
After every headline, there should be some text e.g. a chapter should never begin with a section title.
If you want to reference one part of your text from another part of your text, write something like "see \cref{ch:Example}".
%alternatively use Chapter~\ref{ch:Example}
Take a look at the code to see how to make the number clickable.


\section{Tables}
\label{sec:Example:Tables}
Tables should be \emph{floats} i.e. typeset within "\texttt{\textbackslash{}begin\{table\}}" and "\texttt{\textbackslash{}end\{table\}}" which makes them float to the top of the next page.
Tables should have the full text width.
Further, they should have a meaningful caption placed above them which serves as a standalone explanation and they should also be referenced from the text, like \cref{tb:Example:Tables}.

A visually pleasing table has few lines, unlike tables from e.g. Excel.
A table typically consists of three horizontal lines "$\set{\texttt{top},\texttt{mid},\texttt{bottom}}$\texttt{rule}" and no vertical lines.
Top and bottom rule separate the table from the surrounding text whereas the mid rule separates the header from the data.
All further visual distinction between some rows and columns is achieved by empty space.
For lines, this is possible with the command "\texttt{\textbackslash{}addlinespace}".
The automatic spacing between columns is usually already sufficient.

This example uses the "\texttt{tabularx}" package to make tables.
There are three column types you typically use which are also available in other packages like "\texttt{booktabs}".
Text columns are of type "\texttt{L}" which makes the column left aligned.
Number columns are of type "\texttt{R}" which makes the column right aligned.
Sometimes you use the type "\texttt{C}" to make a column centered.
This is mostly useful for the header row.

\begin{table}[t]
	\caption{This is an example table, the content does not have an actual meaning.
		Note that unlike for figures, the caption is above the content.
		\label{tb:Example:Tables}}
	\begin{tabularx}{\linewidth}{LLRR}
		\toprule
		& Text column & Number column & More numbers \\
		\midrule
		First row & some text & 123\,456 & 1\,234.56 \\
		Second row & some more text & 12\,345\,678 & 7\,839.20 \\
		\addlinespace
		First other row & some text & 654\,321 & 6\,789.10 \\
		Second other row & some more text & 87\,654\,321 & 1\,234.50 \\
		\bottomrule
	\end{tabularx}
\end{table}


\section{Figures}
\label{sec:Example:Figures}
Like tables, Figures should be \emph{floats} i.e. typeset within "\texttt{\textbackslash{}begin\{figure\}}" and "\texttt{\textbackslash{}end\{figure\}}" which makes them float to the top of the next page.
Figures should be centered with "\texttt{\textbackslash{}centering}",
and they should have a meaningful caption which serves as a standalone explanation.
Other than for tables, the cation is placed below the figure.
Further, they should always be referenced from the text, like \cref{fig:Example:figure}.

\begin{figure}[t]
	\centering
	\includegraphics[height=2.5cm]{logos/kitlogo_en_cmyk.pdf}
	\caption{This is the logo of the KIT and is used as an example image in this work.
	Note that unlike for tables, the caption is below the content.
	\label{fig:Example:figure}}
\end{figure}


\section{Math and Numbers}
\label{sec:Example:MathAndNumbers}
A sentence should always end with a full stop even if the sentence ends with maths like this
\[3987^{12}+4365^{12}=4472^{12}\text{~.}\]
Use the latex command "\texttt{\textbackslash,}" as the thousands separator to make numbers like 12\,345 look nicer.
This is especially useful in tables, see \cref{tb:Example:Tables}.

Math formulas look nicer and are easier to read if you use parentheses of varying sizes, take a look at the following two formulas
\begin{align*}
	f&\in\mathcal{O}((n+m)\cdot\log(n))\\
	g&\in\mathcal{O}\big((n+m)\cdot\log(n)\big)\text{~.}
\end{align*}
Try to avoid starting sentences with a math symbol.
The multiplication symbol is a dot typeset with the command "\texttt{\textbackslash{}cdot}" instead of "$*$" or "$\times$".


\section{Citations}
\label{sec:Example:Citations}
Put citation marks whenever you use statements by another person or published by yourself outside of this work.
Put citation marks at the and of a sentence since they distract the reader.
Dibbelt et al. \cite{DSW16} did something.
Dibbelt et al. did something \cite{DSW16}.


\section{General notes}
\label{sec:Example:Notes}
These are just some unrelated tips and tricks.

\paragraph{Git}
Put every sentence in a new line inside the .tex file to make "\texttt{git diff}" and other commands work nicer.

\paragraph{Text}
Avoid enumeration and itemize environments in written text.
Do not use short forms like "don't".
However, common abbreviations like "et al.", "e.g.", and "i.e." should be used.

\paragraph{Floats}
Try to avoid two floats at the begin of the same page, especially if they are of different types.
See \cref{fig:Example:figure} and \cref{tb:Example:Tables} as a counter example.

\paragraph{TODOs}
As you are working on the document, some things may need further
attention again at a later time.\todo{Include results obtained by
  analyzing the Xen crystal with the anti-mass spectrometer, once the
  experiments are completed.}  Then, it might be useful to annotate
the corresponding part with a \texttt{todo}.  To make it really
obvious that something important is missing at some
point. \todo[inline]{You can include an \texttt{inline} note as well.}
Also do not forget to check your thesis for left-over \texttt{todo}s
before printing and handing in.


\section{Examples}
Some random examples.
\begin{theorem}[Euclid's theorem]
	The set of primes $\mathbb{P}$ is not finite.
	\label{thm:Example:Examples:euclid}
\end{theorem}
\begin{proof}
	Suppose that we have a finite set $\mathbb{P}$ which contains all primes and let $a$ be the product of all primes in $\mathbb{P}$.
	Then $a$ and $a+1$ do not share a common factor other than $1$.
	However, this implies that $a+1$ contains a prime factor which is missing in $\mathbb{P}$.
	Therefore, our assumption that $\mathbb{P}$ contains all primes is wrong which implies that no finite set containing all primes can exist.
\end{proof}

\begin{theorem}
	The set of primes $\mathbb{P}$ is not finite.
\end{theorem}
\begin{proofSketch}
	Take a look at the sum over the inverse of all primes
	\[\sum_{p\in\mathbb{P}} \frac{1}{p}\]
	and show that this sum diverges which directly implies that $\mathbb{P}$ cannot be finite.
\end{proofSketch}

\begin{corollary}
	A direct implication of \cref{thm:Example:Examples:euclid} is that there is no largest prime.
\end{corollary}

\begin{algorithm}[H]
	\Input{An integer $n$.}
	\Output{The largest prime factor of $n$.}
	\BlankLine
	$x \longleftarrow 1$\;
	\ForAll{$i$ \With $i\cdot i \leq n$}{
		\While{$n \bmod i = 0$}{
			$n\longleftarrow \frac{n}{i}$\;
			$x\longleftarrow \max(x, i)$\;
		}
	}
	\If{$n > 1$}{
		$x \longleftarrow n$\;
	}
	\Return{$x$}
	\caption{A factorization algorithm.
	\label{alg:Example:Examples:algorithm}}
\end{algorithm}

