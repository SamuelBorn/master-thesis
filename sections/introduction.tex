\chapter{Introduction}
\label{ch:introduction}

\todo[inline]{fluff text}

\section{Motivation}
\label{sec:motivation}

In graph theory, a separator is a subset of vertices whose removal divides the graph into disconnected components of roughly equal size.
The size of these separators significantly affects the performance of numerous algorithms, particularly those utilizing divide-and-conquer approaches.
For instance, Tarjan's and Lipton's foundational work on planar graphs \cite{lipton_applications_1977} demonstrates their utility in optimizing algorithms for many problems like e.g. maximum flow.

Empirical studies suggest road networks have balanced separators on the order of \bigO{n^{1/3}} \cite{dibbelt_customizable_2016}, which is significantly smaller than the \bigO{n^{1/2}} worst-case bound for planar graphs \cite{lipton_separator_1979}.
This is noteworthy given that road networks are can be treated as nearly planar due to their geographic nature (few crossings from overpasses/tunnels).

In road networks, these separators enable the creation of effective node orders, which are critical for the performance of search queries in advanced routing algorithms like CCH.
This thesis seeks to uncover the properties responsible for the presence of such small separators in road networks.
We aim to determine whether these separators stem from inherent graph characteristics, such as limited vertex degrees or sparsity, or from physical real-world features, such as borders, rivers, or a hierarchical structure.
Gaining insight into these properties promises to advance our theoretical understanding and offers practical benefits, such as identifying new applications or generating comparable synthetic graphs.

Road networks represent an intriguing subject for the study of graph separators.
Classical results within graph theory primarily establish balanced separators characterized by asymptotic sizes such as \(\Theta\!\left(1\right)\), common in structures like trees, or \(\Omega\!\left(\sqrt{n}\right)\), e.g. planar graphs and unit disk graphs.
To the best of our knowledge, established graph classes consistently exhibiting separator sizes strictly between these \(\Theta\!\left(1\right)\) and \(\Omega\!\left(\sqrt{n}\right)\) bounds are not prominently featured in the graph theory literature.
This finding positions road networks within this sparsely populated intermediate complexity range, thereby highlighting the compelling nature of investigating their structural properties.

\section{Contribution}
\label{sec:Contribution}

\todo[inline]{todo}

\section{Outline}
\label{sec:overview}

\todo[inline]{todo}
