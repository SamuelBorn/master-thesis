\chapter{Introduction}
\label{ch:introduction}

In this chapter, the motivation for this work is explained. 
We shortly introduce our contribution.
Finally, a brief outline of the contents is given.

\section{Motivation}
\label{sec:motivation}

In graph theory, a separator is a subset of vertices whose removal divides the graph into disconnected components of roughly equal size.
The size of these separators significantly affects the performance of numerous algorithms, particularly those utilizing divide-and-conquer approaches.
For instance, Tarjan's and Lipton's foundational work on planar graphs \cite{lipton_applications_1977} demonstrates their utility in optimizing algorithms for many problems like e.g. maximum flow.

Empirical observations from the Customizable Contraction Hierarchies (CCH) paper indicate that road networks may possess remarkably small separators, on the order of \bigO{n^{1/3}} \cite{dibbelt_customizable_2016}.
This finding contrasts with the \bigO{n^{1/2}} separators typical of planar graphs \cite{lipton_separator_1979}, especially since road networks can be considered as nearly planar.
In road networks, these separators enable the creation of effective node orderings, which are critical for the performance of search queries in advanced routing algorithms like CCH.
This thesis seeks to uncover the properties responsible for the presence of such small separators in road networks.
We aim to determine whether these separators stem from inherent graph characteristics, such as limited vertex degrees or sparsity, or from physical real-world features, such as borders, rivers, or a hierachical sturcutre.
Gaining insight into these properties promises to advance our theoretical understanding and offers practical benefits, such as identifying new applications or generating comparable synthetic graphs.
The exploration of small separators in road networks is particularly fascinating due to the rarity of other naturally occurring graph classes exhibiting separators smaller than those in planar graphs.

\section{Contribution}
\label{sec:Contribution}

\todo[inline]{todo}

\section{Outline}
\label{sec:overview}

\todo[inline]{todo}
