\chapter{Conclusion}
\label{ch:conclusion}

This chapter begins by summarizing the key findings, detailing the results of the various models tested.
Subsequently, it addresses the limitations of our experimental approach and outlines several promising directions for future research that build upon the insights gained.

\section{Summary of Findings}
\label{sec:conclusion:summary}

We demonstrate that the separator size of road networks \(s\) scales with graph size \(n\) according to \(\bigO{n^{0.37}}\).
This exponent is slightly larger than the previously suggested \(\bigO{n^{1/3}}\) \cite{dibbelt_customizable_2016} but remains substantially better than the \(\bigO{n^{1/2}}\) bound for planar graphs.

Our investigation systematically demonstrates the insufficiency of simple, isolated graph properties to explain this behavior.
Models based solely on degree distribution or basic geometric locality consistently produce separators that scale as \bigO{n} or \bigO{n^{1/2}}.
Even considering graphs with a geometric embedding, which is defining feature of road networks, standard models like grids or Delaunay triangulations fail to reproduce the target scaling.
Furthermore, planarizing real road networks has a negligible impact on their separator sizes, suggesting that the non-planarity of these graphs is not a primary factor in the observed small separators.

In contrast, models incorporating a hierarchical structure show significant promise.
Our proposed hierarchical Delaunay generator constitutes an entire parameterized class of graphs.
By carefully tuning its parameters, which govern expansion fractions, points per site, and radii across multiple levels, this model is also capable of replicating the observed \bigO{n^{0.37}} scaling.
Notably, the resulting graph structures often bear a strong visual resemblance to those generated by the physical barrier model.

The most successful synthetic model developed in this thesis, however, is one based on simulating physical barriers using multi-scale Perlin noise.
This approach generates a layered landscape of obstacles at various scales, constraining where vertices can be placed.
Remarkably, graphs generated using this method, which combines noise-based point sampling and a Delaunay triangulation, naturally produce separators that scale approximately as \bigO{n^{0.37}}.
An optional pruning step can be employed to reduce the average degree, but this is not strictly necessary to achieve the desired scaling.
This result is achieved without extensive parameter fine-tuning, suggesting that this model captures a more fundamental generative principle.
Ablation studies further underscore this finding, demonstrating that the full spectrum of noise scales, from large, regional barriers to small, local ones, is crucial for achieving this specific scaling across a wide range of graph sizes.

It is important to qualify, however, that the need for a multi-scale or hierarchical structure is not strictly necessary if one allows for extreme parameter fine-tuning.
The tree-locality model, for instance, could replicate the desired scaling using a highly specific distance-decay function \(f(\text{dist}) = 1/\text{dist}^{3.3}\).
This case serves as an interesting exception, highlighting that specific correlation structures can be enforced without an explicit hierarchy, though this offers less insight into the natural emergence of such properties.

Revisiting the central research question, \enquote{Do small separators in road networks arise from intrinsic graph properties, or from real-world physical barriers?}, our findings provide a nuanced answer.
The evidence strongly suggests that small separators are not a consequence of any single, simple graph property but are an emergent feature of a multi-scale structural organization.
This multi-scale structure can be interpreted as an explicit hierarchy of road types or, perhaps more fundamentally, as the result of physical barriers existing at all geographic scales.
The success of the multi-scale Perlin noise model indicates that the constrained placement of nodes by geography is a powerful generative mechanism.
Road networks are built upon a landscape permeated by obstacles at every scale, from mountain ranges down to local properties.
This forces the graph to be composed of densely connected regions that are themselves sparsely interconnected through well-defined corridors, a structure highly amenable to small separators.
Therefore, we conclude that the small separators in road networks are a direct result of their adaptation to a multi-scale, obstacle-rich environment.

\section{Limitations and Future Work}
\label{sec:conclusion:future_work}

The conclusions presented in this thesis are primarily derived from experimental analysis.
While our results strongly indicate the importance of hierarchy and multi-scale barriers, the inability of simpler, non-hierarchical models to reproduce the desired scaling does not constitute a formal disproof of their potential under different assumptions.
A notable limitation is that our most successful models rely on a final Delaunay triangulation.
The observation that k-Nearest Neighbor graphs built on the same hierarchical point sets fail to produce the correct scaling suggests that the specific connectivity pattern of Delaunay graphs, which allows for both local and some structured long-range connections, plays a significant, yet not fully explored, role.

These limitations and findings open several avenues for future research.
A primary direction is the pursuit of a rigorous theoretical analysis of the multiplicative Perlin noise model.
Such a study could potentially provide a formal derivation for the empirically observed \bigO{n^{0.37}} scaling.
Further work could also seek to improve the efficiency of the generative algorithms.
For instance, the tree-locality generator is hampered by its slow, BFS-based distance calculations; developing an efficient method to sample edges according to weights derived from tree distances would make this model more viable for large-scale tests.
The synthetic generators developed here can serve as a basis for creating realistic, large-scale benchmarks for evaluating not only route planning algorithms but also other algorithms that depend on graphs with small separators.
An important extension of this work would be to adapt the noise generation algorithm to more closely replicate other key road network metrics beyond just separators, such as hop distributions, to create an even more robust synthetic road network generator.
