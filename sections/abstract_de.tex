\section*{\abstractname}

Empirische Beobachtungen legen nahe, dass reale Straßennetzwerke kleine Graphseparatoren aufweisen, deren Größe etwa in der Größenordnung von \bigO{n^{1/3}}, wächst.
Dieses Wachstum ist signifikant langsamer als die Worst-Case-Schranken bekannter Graphklassen, wie die planarer Graphen mit einer Separatorgröße von \bigO{n^{1/2}}.
Die diesem Phänomen zugrunde liegenden strukturellen Eigenschaften sind jedoch bisher nur unzureichend verstanden.
Diese Arbeit untersucht daher systematisch, welche Netzwerkeigenschaften das Vorhandensein kleiner Separatoren erklären.

Unsere Analyse von realen Netzwerkdaten deutet auf ein Wachstum der Separatorgröße von circa \bigO{n^{0.37}} hin.
Wir bewerten den Einfluss verschiedener Grapheigenschaften, indem wir versuchen, dieses Wachstumsverhalten mithilfe synthetisch generierter Graphen zu replizieren.
Die Untersuchung zeigt, dass simplere Eigenschaften wie geringe Dichte oder die Existenz einer Einbettung allein nicht ausreichen, um die in Straßennetzwerken beobachteten Separatorgrößen zu erklären.
Modelle, die allein auf solchen Merkmalen basieren, führen typischerweise zu Separatoren, deren Größe mit \bigO{n^{1/2}} oder schlechter wächst.

Unsere Ergebnisse deuten stattdessen darauf hin, dass die kleinen Separatoren eine direkte Folge einer hierarchischen Struktur sind.
Diese Schlussfolgerung wird durch unsere generativen Modelle gestützt.
Wir stellen fest, dass zwei konzeptionell unterschiedliche Ansätze, einer basierend auf expliziter hierarchischer Konstruktion, der andere auf der Simulation physischer Barrieren mittels mehrskaligem Rauschen, Graphen mit einer Separatorgröße von \bigO{n^{0.37}} erzeugen.
Dieses Ergebnis kommt den empirischen Beobachtungen in Straßengraphen sehr nahe.
Der Erfolg dieser diversen Modelle, die eine hierarchische Organisation erzwingen, legt nahe, dass Hierarchie die entscheidende Eigenschaft ist, die für die kleinen Separatoren in Straßennetzwerken verantwortlich ist.
