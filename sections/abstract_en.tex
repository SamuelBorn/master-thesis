\section*{\abstractname}

Empirical observations suggest that road networks possess small graph separators, scaling approximately as \bigO{n^{1/3}}.
This scaling is substantially smaller than theoretical bounds for major graph classes, notably planar graphs, which are a natural point of comparison due to the near-planar structure of road networks.
The underlying structural properties of road networks responsible for this phenomenon, however, remain poorly understood.
This thesis systematically investigates which network characteristics explain these small separators.

Our analysis of real-world network data indicates a slightly larger separator scaling of approximately \bigO{n^{0.37}}.
We evaluate the impact of several fundamental graph properties, including degree distribution, locality, and geometric embeddings, by attempting to replicate this scaling behavior with synthetic graph models.
The findings demonstrate that simple properties like sparsity or near-planarity alone are insufficient.
Models based on these characteristics consistently yield separators that scale as \bigO{n^{1/2}} or worse.

Instead, our results indicate that small separators are an emergent property of a hierarchical structure.
This conclusion is supported by our generative models.
We find that conceptually different approaches, including a model based on explicit hierarchical construction and another simulating physical barriers with multi-scale noise, can both produce graphs whose separators scale as \bigO{n^{0.37}}, closely matching our empirical findings.
The shared success of diverse models that enforce a hierarchical organization identifies this as the critical property responsible for small separators in road networks.
