%% LaTeX2e class for student theses
%% sections/abstract_en.tex

\section*{\abstractname}

Road networks empirically exhibit remarkably small graph separators, scaling approximately as \(\bigO{n^{0.37}}\).
This behavior is significantly better than the theoretical \(\bigO{n^{1/2}}\) bound for general planar graphs, yet the underlying structural properties responsible for this phenomenon are not well understood.
This thesis systematically investigates which characteristics of road networks give rise to these small separators.

We evaluate the impact of several fundamental graph properties, including degree distribution, diameter, planarity, and hierarchy, by analyzing real-world network data and attempting to replicate the observed separator scaling with synthetic graph models.
Our findings demonstrate that simple properties like sparsity or near-planarity alone are insufficient; models based on these consistently yield separators scaling as \(\bigO{n^{1/2}}\) or worse.
Instead, our results indicate that the small separators are an emergent property of a complex, multi-scale structure.

The most successful model developed in this work simulates the presence of physical barriers at all geographic scales using multiplicative Perlin noise.
Without extensive parameter fine-tuning, this generative approach naturally produces graphs whose separators scale as \(\bigO{n^{0.37}}\), closely matching empirical data.
Ablation studies confirm that a full spectrum of noise scales, analogous to a deep hierarchy of real-world obstacles, is crucial for this result.
