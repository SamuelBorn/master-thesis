\chapter{Preliminaries}
\label{ch:preliminaries}

\section{Graph Theory}
\label{sec:graphtheory}

Road networks can be modeled as graphs.
A graph \(G\) is formally defined as a pair \((V, E)\), where \(V\) represents a finite set of vertices (or nodes) and \(E\) represents a set of edges connecting pairs of vertices.
In many applications, particularly route planning, graphs are augmented with a weight function \(w: E \to \mathbb{R}^+\), assigning a positive real value such as distance or travel time to each edge.
However, for the study of graph separators, the edge weights are typically not considered.
This thesis focuses on simple graphs, meaning graphs without multiple edges between the same pair of vertices and without edges connecting a vertex to itself (loops).
Furthermore, as the concept of separators primarily applies to connectivity, we will consider undirected graphs, where edges represent symmetric relationships.
An edge connecting vertices \(u\) and \(v\) in an undirected graph is denoted as the set \(\{u,v\}\).
A \emph{geometric graph} associates each vertex \(v \in V\) of a graph \(G = (V, E)\) with a unique point \(p\) in a specific geometric space, such as the Euclidean plane \(\mathbb{R}^2\) or the surface of a sphere.

\section{Graph Separators}
\label{sec:graphseparators}

A vertex separator (or simply separator) of a graph \(G = (V, E)\) is a subset of vertices \(S \subseteq V\) whose removal disconnects the graph into two or more components.
More formally, the subgraph induced by \(V \setminus S\), denoted \(G[V \setminus S]\), is disconnected.
For algorithmic applications, particularly divide-and-conquer strategies, balanced separators are crucial.
A separator \(S\) is called \(\alpha\)-balanced, if removing \(S\) partitions the remaining vertices \(V \setminus S\) into disjoint sets \(V_1, V_2, \dots, V_k\) such that no vertex in \(V_i\) is adjacent to a vertex in \(V_j\) for \(i \neq j\), and the size of each resulting component \(V_i\) is bounded: \(|V_i| \le \alpha \cdot |V|\).
A common requirement is \(2/3\)-balancedness, meaning each component contains at most \(2/3\) of the original graph's vertices.
Balancedness ensures that recursive applications of the separator lead to subproblems of substantially smaller size, which is essential for the efficiency of algorithms based on this technique.
The efficiency is often determined by the size of the separator, which is typically expressed as an asymptotic function of the number of vertices \(n = |V|\) in the graph.
