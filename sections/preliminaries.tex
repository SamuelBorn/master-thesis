\chapter{Preliminaries}
\label{ch:preliminaries}

This chapter establishes the theoretical preliminaries that form the basis of our investigation.
We begin by covering fundamental concepts and terminology from graph theory, focusing on the properties relevant to large networks like roads.
Subsequently, we provide a detailed exposition of graph separators, the central concept of this thesis, including their formal definition, the importance of balanced partitions, and the specific algorithms employed in our work to compute them.

\section{Graph Theory}
\label{sec:graphtheory}

Road networks can be modeled as graphs.
A graph \(G\) is formally defined as a pair \((V, E)\), where \(V\) represents a finite set of vertices (or nodes) and \(E\) represents a set of edges connecting pairs of vertices.
In many applications, particularly route planning, graphs are augmented with a weight function \(w: E \to \mathbb{R}^+\), assigning a positive real value such as distance or travel time to each edge.
However, for the purpose of this thesis, the topological structure of the graph is of primary interest, and we will not focus on edge weights.
We will also only consider simple graphs, meaning graphs without multiple edges between the same pair of vertices and without edges connecting a vertex to itself (loops).
Furthermore, as the concept of separators primarily applies to connectivity, we will consider undirected graphs, where edges represent symmetric relationships.
An edge connecting vertices \(u\) and \(v\) in an undirected graph is denoted as the set \(\{u,v\}\).
The neighborhood of a vertex \(v\) is defined as the set of vertices adjacent to \(v\), denoted as \(N : V \to \mathcal{P}(V)\).

A graph \emph{embedding} assigns each vertex \(v \in V\) of a graph \(G = (V, E)\) to a point \(p\) in a specific geometric space, such as the Euclidean plane \(\mathbb{R}^2\) or the surface of a sphere.
We then consider edges as straight line segments connecting the points corresponding to their incident vertices.

Throughout this thesis, the term \emph{graph size} refers specifically to the number of vertices, \(|V|\).
We frequently adopt the notation \(n\) for the number of vertices, \(|V|\), and \(m\) for the number of edges, \(|E|\).
It is worth noting that for many graph classes discussed herein, particularly sparse graphs like road networks, the number of edges \(m\) is asymptotically linear in the number of vertices \(n\).
For planar graphs, a specific bound guarantees this linear growth.
Euler's formula for connected planar graphs states that \(n - m + f = 2\), where \(f\) is the number of faces (regions) defined by the graph embedding.
By observing that each face is bounded by at least three edges (for \(n \ge 3\)) and each edge separates at most two faces, we derive the inequality \(2m \ge 3f\).
Substituting \(f = 2 - n + m\) from Euler's formula into this inequality yields \(m \le 3n - 6\).
This confirms the linear relationship between the number of edges and vertices for planar graphs.
Therefore, the choice of \(n\) as the measure of size generally does not impact asymptotic complexity results for the graphs under consideration.

\section{Graph Separators}
\label{sec:graphseparators}

A vertex separator (or simply separator) of a graph \(G = (V, E)\) is a subset of vertices \(S \subseteq V\) whose removal disconnects the graph into two or more components.
More formally, the subgraph induced by \(V \setminus S\), denoted \(G[V \setminus S]\), is disconnected.
For algorithmic applications, particularly divide-and-conquer strategies, balanced separators are crucial.

Let \(V_1, \dots, V_k\) be the vertex sets corresponding to the connected components of the subgraph \(G[V \setminus S]\).
Most often, the removal of such a separator yields exactly two components (\(k=2\)), as partitioning the graph into a larger number of components generally demands a larger separator.
For a given constant \(\alpha \in (0, 1)\), a separator \(S\) is termed \(\alpha\)-balanced if the size of every resulting component \(V_i\) is bounded.
Specifically, the condition \(|V_i| \leq \alpha |V|\) must hold for all \(i \in \{1, \dots, k\}\).
A simple illustration of a balanced separator is shown in \cref{fig:separator_example_balanced}.
A common requirement is \(2/3\)-balancedness, meaning each component contains at most \(2/3\) of the original graph's vertices.
Balancedness ensures that recursive applications of the separator lead to subproblems of substantially smaller size, which is essential for the efficiency of algorithms based on this technique.

Furthermore, minimizing the size of the separator \(S\) itself is critical for algorithmic performance.
The size of the separator is typically evaluated asymptotically as a function of the number of vertices \(n = |V|\) e.g. \(n^\beta\) for \(\beta \in (0,1)\).

The concept of \emph{recursive} \(\alpha\)-balanced separators extends this idea by ensuring that the property of finding small, balanced separators persists in the resulting subgraphs.
Specifically, after removing an \(\alpha\)-balanced separator \(S\) from \(G\), each induced subgraph \(G[V_i]\) (\({i \in \{1, 2, \dots, k\}}\)) can itself be partitioned using another \(\alpha\)-balanced separator of small size.

\begin{figure}[tbhp]
	\centering
	\begin{subfigure}{0.45\linewidth}
		\centering
		\begin{tikzpicture}[every node/.style={circle, draw, minimum size=0.7cm}]
			\node (1) {1};
			\node (2) [right=of 1] {2};
			\node (3) [right=of 2] {3};
			\node (4) [below=of 1] {4};
			\node (5) [below=of 2, fill=teal!30] {5};
			\node (6) [below=of 3] {6};
			\node (7) [below=of 4] {7};
			\node (8) [below=of 5, fill=teal!30] {8};

			\draw (4) -- (5);
			\draw (5) -- (7);
			\draw (5) -- (2);
			\draw (6) -- (5);
			\draw (7) -- (8);
			\draw (8) -- (6);
			\draw (4) -- (1);
			\draw (7) -- (4);
			\draw (2) -- (3);
			\draw (3) -- (6);
		\end{tikzpicture}
		\caption{\(G\) with separator \(S = \{5, 8\}\)}
	\end{subfigure}
	\hfill
	\begin{subfigure}{0.45\linewidth}
		\centering
		\begin{tikzpicture}[every node/.style={circle, draw, minimum size=0.7cm}]
			\node (1) {1};
			\node (2) [right=of 1] {2};
			\node (3) [right=of 2] {3};
			\node (4) [below=of 1] {4};
			\node (6) [below=of 3] {6};
			\node (7) [below=of 4] {7};

			\draw (4) -- (1);
			\draw (7) -- (4);
			\draw (2) -- (3);
			\draw (3) -- (6);
		\end{tikzpicture}
		\caption{\(G[V \setminus S]\)}
	\end{subfigure}
	\caption{Example of a well balanced separator in a graph. The vertices 5 and 8 form a balanced separator that disconnects the graph into two components.}
	\label{fig:separator_example_balanced}
\end{figure}

To compute separators, various algorithms can be employed.
In this thesis, we primarily utilize InertialFlowCutter \cite{gottesburen_faster_2019}.
This algorithm leverages geometric embeddings, often available for road networks, to compute high-quality node orderings efficiently.
These node orderings serve as the basis for extracting separators from the graph using the method described below \cite{blasius_customizable_2025}.
Employing this combined approach provides a practical pathway to generate separators for our analysis, adapting the use of InertialFlowCutter's output to suit the specific requirements of this work.
Since InertialFlowCutter is specifically designed for the structure of road networks, its performance on the more arbitrary synthetic graphs investigated in this thesis is not guaranteed to be optimal.
To address this potential bias and validate our results, we cross-check our findings using two other established partitioning frameworks: KaHIP \cite{sanders_think_2013} and FlowCutter \cite{hamann_graph_2018}.
Despite its specialization, we observe that for most graph instances examined in this thesis, InertialFlowCutter consistently produces separators of smaller size compared to the other two methods.
A key constraint of InertialFlowCutter, however, is its requirement of a geometric embedding as input.
Consequently, for synthetic graphs generated without an explicit embedding, we rely exclusively on KaHIP and FlowCutter for separator computation.

When using InertialFlowCutter, the resulting node ordering is interpreted as an elimination order for the vertices of the graph \( G = (V, E) \).
Based on this order, a chordal supergraph \( G_C = (V, E \cup F) \) is constructed, where \( F \) represents the fill-in edges.
The chordal supergraph is constructed by processing vertices \(v\) according to their rank.
Fill-in edges are added such that for each vertex \(v\), all its neighbors with a rank greater than \(rank(v)\) form a clique.
An efficient implementation connects the neighbor \(u_{min}\) with the minimum rank among those where \(rank(u_{min}) > rank(v)\) to all other neighbors \(w\) also satisfying \(rank(w) > rank(v)\).
This suffices because the responsibility for adding edges between the remaining pairs of these higher-ranked neighbors \(w\) is effectively delegated to \(u_{min}\).

Afterwards, a tree structure \(T\) can be constructed.
Each node \(v \in V\) selects its parent in the tree as the neighbor \(u\) that appears earliest in the elimination order among all neighbors \(w\) in \( G_C \) with \( \text{rank}(w) > \text{rank}(v) \).
If a node has no neighbors later in the order, it becomes the root.

Separators in the original graph \(G\) can be derived from this tree structure using a traversal algorithm.
The fundamental idea is to identify paths representing non-branching segments of the tree.
Starting from a node \(r\) (representing the current subgraph), the traversal follows a path \(P = (v_1=r, v_2, \dots, v_k)\) downwards, where each node \(v_i\) (\(1 \le i < k\)) has exactly one child \(v_{i+1}\) in the tree.
The path ends at node \(v_k\), which is the first node encountered that does not have exactly one child (i.e., it has zero or multiple children).
The set of vertices on this path, \(S = \{v_1, v_2, \dots, v_k\}\), forms a separator.
Its size is \(k\), the number of nodes on the path.
The traversal algorithm continues recursively into these subtrees.
An overview of this process is illustrated in \cref{fig:extract_separator_from_order}.

\begin{figure}[tbhp]
	\begin{subfigure}{0.3\linewidth}
		\centering
		\begin{tikzpicture}[every node/.style={circle, draw, minimum size=0.7cm}]
			\node (4) {4};
			\node (1) [right=of 4] {1};
			\node (2) [below=of 4] {2};
			\node (3) [right=of 2] {3};

			\draw (4) -- (1) -- (3) -- (2) -- (4);
		\end{tikzpicture}
		\caption{Original graph \(G\)\newline}
	\end{subfigure}
	\hfill
	\begin{subfigure}{0.3\linewidth}
		\centering
		\begin{tikzpicture}[every node/.style={circle, draw, minimum size=0.7cm}]
			\node (4) {4};
			\node (1) [right=of 4] {1};
			\node (2) [below=of 4] {2};
			\node (3) [right=of 2] {3};

			\draw (4) -- (1) -- (3) -- (2) -- (4);
			\draw[dashed] (4) -- (3);
		\end{tikzpicture}
		\caption{Chordal supergraph based on the order}
	\end{subfigure}
	\hfill
	\begin{subfigure}{0.3\linewidth}
		\centering
		\begin{tikzpicture}[every node/.style={circle, draw, minimum size=0.7cm, distance=0.1cm}, node distance=0.5cm and 0.8cm]
			\node (3) [fill=teal!30]{3};
			\node (4) [above=of 3, fill=teal!30] {4};
			\node (1) [below left=of 3] {1};
			\node (2) [below right=of 3] {2};

			\draw (3) -- (1);
			\draw (3) -- (2);
			\draw (3) -- (4);
		\end{tikzpicture}
		\caption{Elimination tree \(T\), Separator \{4,3\} highlighted}
	\end{subfigure}
	\caption{Example Process of deriving a separator from a node order. Node labels in indicate their rank in the node order.}
	\label{fig:extract_separator_from_order}
\end{figure}

To ensure separators yield balanced partitions, the extraction process is refined using a significance threshold based on relative subgraph size.
Child nodes are only considered significant if the subgraph they represent exceeds this threshold (e.g., contains at least 5\% of the nodes in the parent's subgraph).
When tracing a potential separator path \(P = (v_1, \dots, v_k)\) downwards, the path extends from \(v_i\) to \(v_{i+1}\) only if \(v_{i+1}\) is the \emph{single} significant child of \(v_i\).
The path \(S = \{v_1, \dots, v_k\}\) is finalized as the separator upon reaching the first node \(v_k\) that possesses \emph{two or more} significant children.
This ensures separators correspond to meaningful branches in the tree structure concerning substantial graph parts.
\cref{fig:separator_extraction_threshold_example} illustrates an example of this process.

\begin{figure}[tbhp]
	\centering
	\begin{tikzpicture}[every node/.style={circle, draw, minimum size=0.7cm}, node distance=0.2cm and 0.0cm]
		\node (1) [fill = teal!30]{1};
		\node (2) [below left=of 1, fill=teal!30] {2};
		\node (3) [below left=of 2, fill=teal!30] {3};
		\node (4) [below left=of 3, fill=teal!30] {4};
		\node (5) [below left=of 4] {5};
		\node (6) [below right=of 4] {6};
		\node (7) [below right=of 2] {7};

		\node [below = 0cm of 5, draw =none] {\dots};
		\node [below = 0cm of 6, draw =none] {\dots};

		\draw (1) -- (2) -- (3) -- (4) -- (5);
		\draw (2) -- (7);
		\draw (4) -- (6);
	\end{tikzpicture}
	\caption{Separator identification with a significance threshold. The path extends downwards from node 1. At node 2, child 7 represents an insignificant subgraph (below the threshold), so the traversal continues via the single significant child path towards node 3. Node 4 is the first node encountered with two children (5 and 6) that both represent significant subgraphs. Therefore, the process stops here, and the identified separator is S = \{1, 2, 3, 4\}.}
	\label{fig:separator_extraction_threshold_example}
\end{figure}
